\documentclass{article}
\usepackage{amsmath}
\begin{document}
\title{Notes on the Rothermel Model}
\author{By: Jeremy Benik}
\maketitle
The paper starts out discussing how fire is spread with regards to preheating the fuel, dehydrating it, and getting to ignition temperature. It then discusses how during the preheating phase, the fuel temp exceeds the air temperature which indicates convective heating or direct flame contact does not occur until the fire front reaches the particle. The paper then starts to go into the multiple equations and simplifications
\section{Conception on mathematical model}
\indent This model was developed from a strong theoretical base to make its application as wide as possible. this base was supplied by Fransden (1971) who applied the conservation of energy principle to a unit volume of fuel ahead of an advancing fire in a homogeneous fuel bed which led to: 
\begin{equation} 
\label{1}
R = \frac {I_{xig} + \int_{-\infty}^{0} (\frac {\partial I_{z}} {\partial z})_{z_c}\,dx }{\rho_{be} * Q_{i_g}}
\end{equation}
\newline
Where R = Quasi-steady rate of spread, ft./min \\

\noindent I$_{xig}$ = horizontal heat flux absorbed by a unit volume of fuel at the time of ignition, $B.t.u/ft.^2$ -min \newline

\noindent $\rho_{be}$ = Effective bulk density(the amount of fuel per unit volume of the fuel bed raised to ignition ahead of the advancing fire), lb./ft.3 \\

\noindent $Q_{ig}$ = heat of preignition (the heat required to bring a unit weight of fuel to ignition),. B.T.U./lb \\

\noindent $(\frac {\partial {I_z}} {\partial z})_{z_c}$ = The gradient of the vertical intensity evaluated at a plane at a constant depth, $z_c$, of the fuel bed, $B.t.u./ft.^3 -min$
	
To be able to solve equation \ref{1}, each term has to be evaluated and determine experimental and analytical methods of evaluation. 

\section{Heat Required for Ignition}
\indent The heat required for ignition is dependent upon (a) ignition temperature, (b) moisture content of the fuel, and (c) amount of fuel involved in the ignition process. To account for these 



	\end{document}